\documentclass{article}

\usepackage[paper=letterpaper,margin=2.5cm]{geometry} % Set Margins

%% Math and math fonts
\usepackage{amsmath, amsthm, amssymb, amsfonts}
\usepackage{bbm} % for \mathbbm{1}

% date
\usepackage[nodayofweek]{datetime}

% Color
\usepackage{color, xcolor}

% Misc
\usepackage{environ}  % \collect@body in asmmath
\usepackage{graphicx} % \includegraphics options
\usepackage{mdframed} % text boxes
\usepackage{indentfirst} % Indent first paragraph after section header
\usepackage[shortlabels]{enumitem} % Control enumerate items with [(a)]
\usepackage{comment} % Comments
\usepackage{fancyhdr} % Headers and footers

% Tables
\usepackage{array}

% Sub-figures and figure placement
\usepackage{caption}
\usepackage{subcaption}
\usepackage{float} 

% Graphing
\usepackage{pgfplots}
\pgfplotsset{compat=1.17}
\usepackage{tikz}

% Title Placement
\usepackage{titling}
\setlength{\droptitle}{-6em}

%set indent to 
\setlength{\parindent}{0pt}

% Hyper refs
\usepackage{hyperref}
\hypersetup{
    colorlinks=true,
    linkcolor=blue,
    urlcolor  = blue,
    filecolor=magenta,      
    urlcolor=blue,
    citecolor = blue,
    anchorcolor = blue
}

% % Citation management
\usepackage{natbib}
\bibliographystyle{abbrvnat}
\setcitestyle{authordate,open={(},close={)}}

\pagestyle{fancy}

\usepackage[paper=letterpaper,margin=2.5cm]{geometry} % Set Margins

%% Math and math fonts
\usepackage{amsmath, amsthm, amssymb, amsfonts}
\usepackage{bbm} % for \mathbbm{1}

% date
\usepackage[nodayofweek]{datetime}

% Color
\usepackage{color, xcolor}

% Misc
\usepackage{environ}  % \collect@body in asmmath
\usepackage{graphicx} % \includegraphics options
\usepackage{mdframed} % text boxes
\usepackage{indentfirst} % Indent first paragraph after section header
\usepackage{comment} % Comments
\usepackage{fancyhdr} % Headers and footers

% Tables
\usepackage{array}

% Sub-figures and figure placement
\usepackage{caption}
% \usepackage{subcaption}
\usepackage{float} 

% Graphing
\usepackage{pgfplots}
\pgfplotsset{compat=1.17}
\usepackage{tikz}

% Title Placement
\usepackage{titling}
\setlength{\droptitle}{-6em}

%set indent to 
\setlength{\parindent}{0pt}

% Hyper refs
\usepackage{hyperref}
\hypersetup{
    colorlinks=true,
    linkcolor=blue,
    urlcolor  = blue,
    filecolor=magenta,      
    urlcolor=blue,
    citecolor = blue,
    anchorcolor = blue
}

% % Citation management
\usepackage{natbib}
\bibliographystyle{abbrvnat}
\setcitestyle{authordate,open={(},close={)}}

\newcolumntype{M}{>{$}c<{$}} % Define a new column type for math mode


% ----------------------------------------
% TITLE
% ----------------------------------------

\pagestyle{fancy}

\lhead{Creel}
\chead{Review}
\rhead{AMES}

\title{AMES Class Notes -- Week 8, Monday: Linear Algebra}
\author{Andie Creel}

\begin{document}
\maketitle

\section{Intro}

Consider the system of equations: 
\begin{align}
    x &= ay + bz + u \\
    y &= cx + dz + w \\
    z &= fx + gy + v
\end{align}

You could solve this system of three equations for the three unknowns $x,y,z$. But it would be tedious! Let's rearrange it so that we only have to solve for one unknown by using linear algebra. 

\begin{align}
    1x - ay - bz &= u \\
    -cx + 1y - dz &= w \\
    -f x -gy +1z &= v 
\end{align}

This can be rewritten as vectors and matrices. 
\begin{align}
    \begin{bmatrix}
        u \\
        w \\
        v
    \end{bmatrix} = 
    \begin{bmatrix}
        1 & -a & -b \\
        -c & 1 & -d \\
        -f &  -g & 1 \\
    \end{bmatrix} 
    \begin{bmatrix}
        x \\
        y\\
        z
    \end{bmatrix}
\end{align}

Which we can rewrite with vectors and matrices as, 
\begin{align}
    \Vec{N} = \mathbf{M} \Vec{X}
\end{align}

A reoccurring question: How can you write out a system of equations, and how can you rewrite it as a matrix?

\section{Matrices}
The first subscript $i$ is the row and the second subscript $j$ is the column.
\begin{align}
    A = \begin{bmatrix}
        a_{11} & a_{12} & a_{13} & ... & a_{1j}\\
        a_{21} & a_{22} & a_{23} & ... & a_{2j}\\
        ... & ... & ... & ... & ... \\
        a_{i1} & a_{i2} & a_{i3} & ... & a_{ij}\\
    \end{bmatrix}
\end{align}

This is an $(i \times j)$ matrix. Always keep track of your dimensions. 

\section{Addition and Subtraction} 
Consider two $(2 \times 2)$ matrices 

\begin{align}
    A = \begin{bmatrix}
            a_{11} & a_{12} \\
            a_{21} & a_{22}
        \end{bmatrix}, 
    C = \begin{bmatrix}
            c_{11} & c_{12} \\
            c_{21} & c_{22}
        \end{bmatrix}
\end{align}

We can add these matrices because they're of the same dimensions. 

\begin{align}
    A + C = \begin{bmatrix}
                a_{11} + c_{11} & a_{12} + c_{12} \\
                a_{21} + c_{21} & a_{22} + c_{22}
            \end{bmatrix}
\end{align}

Subtraction works the same way.\\

\textbf{Notes:} Addition and subtraction are done element by element. You add all the elements in the [1,1] position, then all the elements in the [1, 2] position, so on and so forth. Notice that to add two matrices together, they must have the same dimensions. 

\section{Scalar Multiplication}
You could "scale" a whole matrix with a scalar. Consider the $2 \times 2$ matrix, A. 

\begin{align}
    A = \begin{bmatrix}
            a_{11} & a_{12} \\
            a_{21} & a_{22}
        \end{bmatrix}
\end{align}

You could scale the matrix A by 10, 
\begin{align}
    10 \times A = \begin{bmatrix}
            10 a_{11} & 10 a_{12} \\
            10 a_{21} & 10 a_{22}
        \end{bmatrix}
\end{align}


\section{Matrix Multiplication}
If we have matrix $B$ that's $(m \times n)$ and matrix $D$ that's $(n \times p)$ then the matrix $B \times M$ is going to be $(m \times p)$. \textbf{Order matters} when multiplying matrices. $M \times B$ is NOT conformable, you cannot compute that matrix!! \\

Consider the matrices $A (2 \times 2)$  and $C (2 \times 2)$ again 
\begin{align}
    A = \begin{bmatrix}
            a_{11} & a_{12} \\
            a_{21} & a_{22}
        \end{bmatrix}, 
    C = \begin{bmatrix}
            c_{11} & c_{12} \\
            c_{21} & c_{22}
        \end{bmatrix}
\end{align}

Now,  $A \times C$. Matrix multiplication requires row multiplied by columns 
\begin{align}
    A \times B = \begin{bmatrix}
        a_{11} c_{11} + a_{12} c_{21}  & a_{11} c_{12} + a_{12} c_{22} \\
        a_{21} c_{11} + a_{22} c_{21}  & a_{21} c_{12} + a_{22} c_{22} 
    \end{bmatrix}
\end{align}

Consider a vector 
\begin{align}
    x = \begin{bmatrix}
            M \\
            N
        \end{bmatrix}
\end{align}

Pre-multiply $x$ by $A$. 
\begin{align}
    \underset{(2 \times 2)}{A} \underset{(2 \times 1)}{x} &= \begin{bmatrix}
            M \\
            N
        \end{bmatrix} 
        \begin{bmatrix}
            a_{11} & a_{12} \\
            a_{21} & a_{22}
        \end{bmatrix} \\
       &= \begin{bmatrix}
           a_{11} M + a_{12}N \\
           a_{21}M + a_{22}N
       \end{bmatrix} 
\end{align}

There is no equivalent to division in matrix algebra. However, you can invert a matrix, which we will get to. 

\section{Transpose}
When you transpose a matrix, the first row becomes the first column. 
\begin{align}
    A = \begin{bmatrix}
        2 & 7 \\
        1 & 3
    \end{bmatrix}\\
    A^{T} = \begin{bmatrix}
        2 & 1 \\
        7 & 3
    \end{bmatrix}
\end{align}

Transpose is important if you're interested in a matrix multiplied by itself! 
\begin{align}
    A^2 = A^{T}A 
\end{align}
To square, you always pre-multiply the original by the transpose. This keeps the matrices conformable. 

\begin{align}
    \underset{(7 \times 3)}{B^2} = \underset{(3 \times 7)}{B^T} \underset{(7 \times 3)}{B} = \underset{ (3 \times 3)}{C}
\end{align}

\textbf{Transpose is not the inverse!!! } Which is a common mistake. 


\section{Names of elements of a matrix}
\begin{itemize}
    \item Diagonal: the elements all the diagonal (top left corner to the lower right)
    \item Lower Triangular matrix: elements under the diagonal 
    \item Upper triangular matrix: elements above the diagonal
\end{itemize}


\section{Identify Matrix}

The identity matrix have similar properties to 1. 
\begin{align}
    I = \begin{bmatrix}
        1 & 0 &  ... \\
        0 & 1 & ... \\
        ... & \\
        0 & 0 & ... & 1
    \end{bmatrix}
\end{align}
There are 1s along the diagonal. An important characteristic of the identity matrix is that for any A
\begin{align}
    \underset{(2 \times 2)} A^{-1}  \underset{(2 \times 2)} A = \begin{bmatrix}
                    1 & 0 \\
                    0 & 1
                \end{bmatrix}
\end{align}

Any matrix pre-multiplied by it's inverse will be an identity matrix. 

\section{Rank}
Rank is the largest unique square matrix you can have inside a matrix.\\

The key here is knowing if a row is unique.  For instance 
\begin{align}
    \begin{bmatrix}
        2 & 4 \\
        1 & 2
    \end{bmatrix}
\end{align}
Only has 1 unique row because the first row is the second row multiplied by 2. 



\end{document}