\documentclass{article}

\usepackage[paper=letterpaper,margin=2.5cm]{geometry} % Set Margins

%% Math and math fonts
\usepackage{amsmath, amsthm, amssymb, amsfonts}
\usepackage{bbm} % for \mathbbm{1}

% date
\usepackage[nodayofweek]{datetime}

% Color
\usepackage{color, xcolor}

% Misc
\usepackage{environ}  % \collect@body in asmmath
\usepackage{graphicx} % \includegraphics options
\usepackage{mdframed} % text boxes
\usepackage{indentfirst} % Indent first paragraph after section header
\usepackage[shortlabels]{enumitem} % Control enumerate items with [(a)]
\usepackage{comment} % Comments
\usepackage{fancyhdr} % Headers and footers

% Tables
\usepackage{array}

% Sub-figures and figure placement
\usepackage{caption}
\usepackage{subcaption}
\usepackage{float} 

% Graphing
\usepackage{pgfplots}
\pgfplotsset{compat=1.17}
\usepackage{tikz}

% Title Placement
\usepackage{titling}
\setlength{\droptitle}{-6em}

%set indent to 
\setlength{\parindent}{0pt}

% Hyper refs
\usepackage{hyperref}
\hypersetup{
    colorlinks=true,
    linkcolor=blue,
    urlcolor  = blue,
    filecolor=magenta,      
    urlcolor=blue,
    citecolor = blue,
    anchorcolor = blue
}

% % Citation management
\usepackage{natbib}
\bibliographystyle{abbrvnat}
\setcitestyle{authordate,open={(},close={)}}

\pagestyle{fancy}

\usepackage[paper=letterpaper,margin=2.5cm]{geometry} % Set Margins

%% Math and math fonts
\usepackage{amsmath, amsthm, amssymb, amsfonts}
\usepackage{bbm} % for \mathbbm{1}

% date
\usepackage[nodayofweek]{datetime}

% Color
\usepackage{color, xcolor}

% Misc
\usepackage{environ}  % \collect@body in asmmath
\usepackage{graphicx} % \includegraphics options
\usepackage{mdframed} % text boxes
\usepackage{indentfirst} % Indent first paragraph after section header
\usepackage[shortlabels]{enumitem} % Control enumerate items with [(a)]
\usepackage{comment} % Comments
\usepackage{fancyhdr} % Headers and footers

% Tables
\usepackage{array}

% Sub-figures and figure placement
\usepackage{caption}
\usepackage{subcaption}
\usepackage{float} 

% Graphing
\usepackage{pgfplots}
\pgfplotsset{compat=1.17}
\usepackage{tikz}

% Title Placement
\usepackage{titling}
\setlength{\droptitle}{-6em}

%set indent to 
\setlength{\parindent}{0pt}

% Hyper refs
\usepackage{hyperref}
\hypersetup{
    colorlinks=true,
    linkcolor=blue,
    urlcolor  = blue,
    filecolor=magenta,      
    urlcolor=blue,
    citecolor = blue,
    anchorcolor = blue
}

% % Citation management
\usepackage{natbib}
\bibliographystyle{abbrvnat}
\setcitestyle{authordate,open={(},close={)}}

% ----------------------------------------
% TITLE
% ----------------------------------------

\pagestyle{empty}

\lhead{Creel}
\chead{Week Six}
\rhead{AMES}

\title{AMES Class Notes -- Week Six, Day 1: Integrals}
\author{Andie Creel}

\begin{document}
\maketitle


\section{Reimann Sums}
An integral is the area under a curve. A reimann sum approximates this area by using rectangles of different heights whose width gets smaller and smaller (\textit{ie} width goes to zero).\\

Let $A(\cdot)$ be an area function. $A$ returns the area under a curve. \\

Recall that small changes in $x$ are equal to $\epsilon$ but also can be written as $\partial x$, $\Delta x = \epsilon = \partial x$.

\begin{align}
    f(a) = \frac{A(a+\epsilon)-A(a)}{\epsilon}
\end{align}

If we take the limit of this function as $\epsilon \to 0$, we get the definition of a derivative 
\begin{align}
    \lim_{\epsilon \to 0} f(a) = \frac{A(a+\epsilon)-A(a)}{\epsilon} \implies\\
    A'(x) = \frac{\partial A}{\partial x} \implies \\
    A'(x) \partial x = \partial A \implies \\
    A'(x) \epsilon = \partial A 
\end{align}
where all of these steps come from knowing $\epsilon = \partial x$ and treating a derivative like a fraction. What is $A'(x)$? If $A(\cdot)$ is the area under the curve, then the derivative is the curve which is $f(x)$ therefore $A'(x) = f(x)$. 

Therfore, a change in area can be written as
\begin{align}
    \partial A = f(x) \epsilon
\end{align}
which is just a rectangle where $f(x)$ is the height and $\epsilon$ is the width. \\

We can rewrite this as 
\begin{align}
    \partial A =  \sum_{x=a}^{b}f(x) \epsilon.
\end{align}
if you want to want to find the area under the curve from the point where $x =a $ to the point where $x = b$. This is a sum of rectangles. \\

Integration is just fancy summation! So, as $\epsilon$ gets really small and we move from discrete changes to arbitrarily small changes (smooth). We can then write this as 
\begin{align}
    \partial A =  \int_{a}^{b} f(x) \partial x.
\end{align}

\section{Additively separable}
Consider a function 
\begin{align}
    f(x) = h(x) + g(x)
\end{align}
where $f(x)$ can be separated into two simpler functions. For instance, any polynomial would take this form (\textit{e.g.} $f(x) = 3x^2 + 6x$). \\

The function $A(\cdot)$ is still returning the area under the curve $f(x)$. Then, we can write the function as 

\begin{align}
    dA = \int h(x) + g(x) dx
\end{align}

Because the integral is a linear operator, we do not need to worry about Jensen's inequality and we can pass the integral sign through the addition 
\begin{align}
    dA = \int h(x) dx + \int g(x) dx
\end{align}

\section{Considering constants}
Remember that when you take the derivative of a constant, it's zero. When you integrate a function, you will not be able to get the constant back out (you don't know what the constant is),

\begin{align}
    F(x) + C = \int f(x) dx.
\end{align}
When ever you take an integral, remember to add on your unidentified constant $C$. \\

If you're doing a \textbf{definite} integral meaning you have the bounds $a$ and $b$ you do NOT need to worry about the unidentified constant $C$ because it will difference out. However, if you are taking an \textbf{indefinite} integral then we don't know the bounds and so we do need to keep track of the unidentified constant.\\

\section{Example }

Consider this the function $f(x) = 3$, 
\begin{align}
    F(x) &= \int f(x) dx\\
    &=  \int 3 dx\\
    &= 3 \int dx \\
    &= 3 \int 1 dx\\
    &= 3x + C
\end{align}

\section{Derivative rules to remember}

\begin{align}
    f(x) = X^n \implies f'(x) = Nx^{N - 1}\\
    f(x) = e^x \implies f'(x) = e^x
\end{align}


\end{document}