\documentclass{article}

\usepackage[paper=letterpaper,margin=2.5cm]{geometry} % Set Margins

%% Math and math fonts
\usepackage{amsmath, amsthm, amssymb, amsfonts}
\usepackage{bbm} % for \mathbbm{1}

% date
\usepackage[nodayofweek]{datetime}

% Color
\usepackage{color, xcolor}

% Misc
\usepackage{environ}  % \collect@body in asmmath
\usepackage{graphicx} % \includegraphics options
\usepackage{mdframed} % text boxes
\usepackage{indentfirst} % Indent first paragraph after section header
\usepackage[shortlabels]{enumitem} % Control enumerate items with [(a)]
\usepackage{comment} % Comments
\usepackage{fancyhdr} % Headers and footers

% Tables
\usepackage{array}

% Sub-figures and figure placement
\usepackage{caption}
\usepackage{subcaption}
\usepackage{float} 

% Graphing
\usepackage{pgfplots}
\pgfplotsset{compat=1.17}
\usepackage{tikz}

% Title Placement
\usepackage{titling}
\setlength{\droptitle}{-6em}

%set indent to 
\setlength{\parindent}{0pt}

% Hyper refs
\usepackage{hyperref}
\hypersetup{
    colorlinks=true,
    linkcolor=blue,
    urlcolor  = blue,
    filecolor=magenta,      
    urlcolor=blue,
    citecolor = blue,
    anchorcolor = blue
}

% % Citation management
\usepackage{natbib}
\bibliographystyle{abbrvnat}
\setcitestyle{authordate,open={(},close={)}}

\pagestyle{fancy}

\usepackage[paper=letterpaper,margin=2.5cm]{geometry} % Set Margins

%% Math and math fonts
\usepackage{amsmath, amsthm, amssymb, amsfonts}
\usepackage{bbm} % for \mathbbm{1}

% date
\usepackage[nodayofweek]{datetime}

% Color
\usepackage{color, xcolor}

% Misc
\usepackage{environ}  % \collect@body in asmmath
\usepackage{graphicx} % \includegraphics options
\usepackage{mdframed} % text boxes
\usepackage{indentfirst} % Indent first paragraph after section header
\usepackage{comment} % Comments
\usepackage{fancyhdr} % Headers and footers

% Tables
\usepackage{array}

% Sub-figures and figure placement
\usepackage{caption}
% \usepackage{subcaption}
\usepackage{float} 

% Graphing
\usepackage{pgfplots}
\pgfplotsset{compat=1.17}
\usepackage{tikz}

% Title Placement
\usepackage{titling}
\setlength{\droptitle}{-6em}

%set indent to 
\setlength{\parindent}{0pt}

% Hyper refs
\usepackage{hyperref}
\hypersetup{
    colorlinks=true,
    linkcolor=blue,
    urlcolor  = blue,
    filecolor=magenta,      
    urlcolor=blue,
    citecolor = blue,
    anchorcolor = blue
}

% % Citation management
\usepackage{natbib}
\bibliographystyle{abbrvnat}
\setcitestyle{authordate,open={(},close={)}}

\newcolumntype{M}{>{$}c<{$}} % Define a new column type for math mode


% ----------------------------------------
% TITLE
% ----------------------------------------

\pagestyle{fancy}

\lhead{Creel}
\chead{Linar Algebra, cont}
\rhead{AMES}

\title{AMES Class Notes -- Week 9, Monday: Linear Algebra, cont}
\author{Andie Creel}

\begin{document}
\maketitle

\section{Squaring matrices}

Consider a matrix $\underset{N \times 1}{V}$ that you'd like to square. 

\begin{align}
    V^2 &= \underset{1 \times N} V^T  \underset{N \times 1}V\\
    &= \underset{ 1 \times 1} Z
\end{align}

Sometimes you'll have a weighting matrix, $\underset{ N \times N}M$

\begin{align}
   \underset{1 \times N} V^T \underset{ N \times N}M  \underset{N\times 1}V = \underset{ 1 \times 1}  Z   
\end{align}


\section{Idempotent matrix}
Defn: A matrix that when squared equals itself 
\begin{align}
    Z^T Z = Z
\end{align}

\section{Kronecker product}

Kronecker products expand out. You see this is computer algorithms and data management problems. It's a common trick. \\

Let
\begin{align}
    A = \begin{bmatrix}
            a_{11} & a_{12} \\
            a_{21} & a_{22}
        \end{bmatrix} \\
    Z = \begin{bmatrix}
        Z_1 \\
        Z_3 
    \end{bmatrix}\\
    A \otimes Z = \begin{bmatrix}
                    a_11 z_1 & a_12 z_1 \\
                    a_11 z_2 & a_12 z_2 \\
                    a_21 z_1 & a_22 z_1 \\
                    a_21 z_2 & a_22 z_2
                \end{bmatrix}
\end{align}

\section{Trace}
Defn: the product of all diagonal elements. The trace of $A$ is $a_{11} * a_{22}$

\section{Linear regression}
Consider the equation estimating equation 
\begin{align}
    y = a + bx + cz + \epsilon
\end{align}

We can rewrite this as

\begin{align}
    \underset{N \times 1} Y = \underset{N \times K } X 
                              \underset{K \times 1} \beta +  \underset{ N \times 1} \epsilon\\
\end{align} 
where 
\begin{align}
    \beta = \begin{bmatrix}
        a\\
        b\\
        c
    \end{bmatrix}\\
    X = \begin{bmatrix}
        1 & x_1& z_1\\
        1 & x_2 & z_2\\
        \vdots & ...
    \end{bmatrix}\\
    Y = \begin{bmatrix}
        y_1\\
        y_2\\
        \vdots
    \end{bmatrix}
\end{align}

\textbf{Goal:} Solve for $\beta$ by minimizing the sum of square errors. \\

Solve for the error term:
\begin{align}
    \underset{N \times 1} Y - \underset{N \times K}{X \beta} = \underset{N \times 1} \epsilon
\end{align}

Square the error term:
\begin{align}
    \underset{1 \times N} \epsilon^T  \underset{N \times 1}\epsilon &= 
        \underset{1 \times N}{(Y - X \beta)^T} \underset{N \times 1}{(Y - X \beta)} \\
    &= \underset{1 \times N}Y^T \underset{N \times 1} Y - \underset{1 \times K} \beta^T \underset{K \times N} X^T \underset{N \times 1} Y - \underset{1 \times N} Y^T \underset{N \times K} X \underset{K \times 1}\beta + \underset{1 \times K}\beta^T \underset{K \times N}X^T \underset{N \times K}X \underset{K \times 1}\beta\\
    &= \underset{1 \times 1}{Y^T Y} - \underset{1 \times 1}{2 \beta^T X^T Y} + \underset{1 \times 1}{\beta^T X^T X \beta}
\end{align}

The squared error term is a scalar $\underset{(1 \times 1)}{\epsilon^T \epsilon}$. \\

We want to minimize the squared error term by choosing $\beta$. To do so, take derivative of squared error and set equal to zero, solve for $\beta$. 

\begin{align}
    \frac{\partial \epsilon^T  \epsilon }{ \partial \beta} = 0 - \underset{K \times 1}{2X^T Y} + \underset{K \times 1}{2 X^T X \beta} = 0 \implies \\
    2X^T Y = 2 X^T X \beta \implies \\
    \beta = {(X^T X)}^{-1} X^T Y
\end{align}

Where ${(X^T X)}^{-1}$ is an inversion (because we cannot divide matrices). 

\section{Life cycle assessment example}

Let's consider and input output table (Ag, Transportation, Manufactured)
\begin{align}
    A = \underset{ 3 \times 3}{\begin{bmatrix}
         & A & T & M \\
         A \\
         T\\
         M
    \end{bmatrix}}
\end{align}

And our final demand (the demand for goods by consumers) 
\begin{align}
    d = \underset{3 \times 1}{\begin{bmatrix}
        A \\
        T\\
        M
    \end{bmatrix}}
\end{align}

Total amount of goods $X$ is, 
\begin{align}
    \underset{3 \times 1}{X} = \underset{3 \times 3}{A} \underset{3 \times 1}{X} + \underset{3 \times 1}{d}
 \end{align}
Solve for X by multiplying with the identity matrix 
\begin{align}
    \underset{3 \times 3}{I} X - AX = d\\
    (I - A) X = d\\
    \implies (I - A)^{-1}d = X
\end{align}

Total amount of goods is different than final demand because we need input good for the final good. A bunch of intermediate products are required to make a computer. 


\section{R and Excel}
Watch the video! Couple of notes \\
- to invert a matrix in R you use the solve() command


\end{document}