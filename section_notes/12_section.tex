\documentclass{article}

\usepackage[paper=letterpaper,margin=2.5cm]{geometry} % Set Margins

%% Math and math fonts
\usepackage{amsmath, amsthm, amssymb, amsfonts}
\usepackage{bbm} % for \mathbbm{1}

% date
\usepackage[nodayofweek]{datetime}

% Color
\usepackage{color, xcolor}

% Misc
\usepackage{environ}  % \collect@body in asmmath
\usepackage{graphicx} % \includegraphics options
\usepackage{mdframed} % text boxes
\usepackage{indentfirst} % Indent first paragraph after section header
\usepackage[shortlabels]{enumitem} % Control enumerate items with [(a)]
\usepackage{comment} % Comments
\usepackage{fancyhdr} % Headers and footers

% Tables
\usepackage{array}

% Sub-figures and figure placement
\usepackage{caption}
\usepackage{subcaption}
\usepackage{float} 

% Graphing
\usepackage{pgfplots}
\pgfplotsset{compat=1.17}
\usepackage{tikz}

% Title Placement
\usepackage{titling}
\setlength{\droptitle}{-6em}

%set indent to 
\setlength{\parindent}{0pt}

% Hyper refs
\usepackage{hyperref}
\hypersetup{
    colorlinks=true,
    linkcolor=blue,
    urlcolor  = blue,
    filecolor=magenta,      
    urlcolor=blue,
    citecolor = blue,
    anchorcolor = blue
}

% % Citation management
\usepackage{natbib}
\bibliographystyle{abbrvnat}
\setcitestyle{authordate,open={(},close={)}}

\pagestyle{fancy}

\usepackage[paper=letterpaper,margin=2.5cm]{geometry} % Set Margins

%% Math and math fonts
\usepackage{amsmath, amsthm, amssymb, amsfonts}
\usepackage{bbm} % for \mathbbm{1}

% date
\usepackage[nodayofweek]{datetime}

% Color
\usepackage{color, xcolor}

% Misc
\usepackage{environ}  % \collect@body in asmmath
\usepackage{graphicx} % \includegraphics options
\usepackage{mdframed} % text boxes
\usepackage{indentfirst} % Indent first paragraph after section header
\usepackage{comment} % Comments
\usepackage{fancyhdr} % Headers and footers

% Tables
\usepackage{array}

% Sub-figures and figure placement
\usepackage{caption}
% \usepackage{subcaption}
\usepackage{float} 

% Graphing
\usepackage{pgfplots}
\pgfplotsset{compat=1.17}
\usepackage{tikz}

% Title Placement
\usepackage{titling}
\setlength{\droptitle}{-6em}

%set indent to 
\setlength{\parindent}{0pt}

% Hyper refs
\usepackage{hyperref}
\hypersetup{
    colorlinks=true,
    linkcolor=blue,
    urlcolor  = blue,
    filecolor=magenta,      
    urlcolor=blue,
    citecolor = blue,
    anchorcolor = blue
}

% % Citation management
\usepackage{natbib}
\bibliographystyle{abbrvnat}
\setcitestyle{authordate,open={(},close={)}}

\newcolumntype{M}{>{$}c<{$}} % Define a new column type for math mode


% ----------------------------------------
% TITLE
% ----------------------------------------

\pagestyle{fancy}

\lhead{Creel}
\chead{Section }
\rhead{AMES}

\title{AMES Week 12 Section - pset 5}
\author{Andie Creel}

\begin{document}
\maketitle

\section{Q1 on PSet 5}
In fisheries management it is common to assume that fish mortality is the sum of two constant instantaneous per capita mortality rates; “natural” mortality, $m$, and fishing mortality, $f $, (natural mortality is everything that is not fishing mortality). 

\subsection{a}
\textit{Consider a shellfish population, N (t). \underline{Assume a constant level of reproduction per unit time}, $A$, independent of the current population (assume that larva drift in from a larger external population due to currents), write an ordinary differential equation showing the change in the fish stock over time. }\\

\textbf{What is an ODE}? It's the derivative of some stock of interest with respect to time, 
\begin{align*}
    \frac{\partial N(t)}{ \partial t} = \text{births minus deaths}\\
    = A - (m +f)N
\end{align*}
Note that $A$ has a unit of constant reproduction per unit of \textit{time}. This is different than a lot of \textit{birth rates} we've seen in class. 

\subsection{b}
\textit{Solve the ordinary differential equation to get $N(T)$, where T is an arbitrary future time.}\\

\textbf{Solution intuition:} We need to get the population level in time period $T$, which is $N(T)$. To get this, we could integrate the \textit{changes} in population through time to get $N(T)$. \\

Problem set up:
\begin{align*}
    \frac{\partial N(t)}{\partial t} = A - (m +f)N \\
\end{align*}
Separate the variables 
\[ \frac{1}{A - (m+f)N} d N = 1 dt\]
Integrate both sides from $N(0)$ to $N(T)$
\[\int_{N(0)}^{N(T)} \frac{1}{A - (m+f)N} d N = \int_{N(0)}^{N(T)} 1 dt\]

You will need to do integration by substitution here. Then evaluate the integral from $N(0)$ to $N(T)$ and solve for $N(T)$. Assume $N(0)$ is known. 


\subsection{c}
\textit{Find the equilibrium solution for $N$.}\\

\textbf{Solution intuition:} What level of $N(t)$ leads to the population to in equilibrium \textit{i.e.} not changing? When $\frac{dN(t)}{dt} = 0 $

\begin{align*}
   \frac{dN(t)}{dt} = A - (m +f)N = 0 
\end{align*}

Solve for $N^*$.

\subsection{d}
\textit{What is the sensitivity of the equilibrium to fishing mortality?}\\

\textbf{Solution intuition:} What is the equilibrium? $N^*$ from part c. What sensitivity mean in math? How $N^*$ changes with respect to fishing mortality, $f$. Therefore, what you want to look at is $\frac{d N^*}{d f}$

\section{Q2 on PSet 5}
Let $y(t)$ be the reserves of oil left in an oil well at time $t$. Suppose extraction takes place at a \textit{constant continuous rate per unit of time} of $\alpha$. 

\subsection{a}
\textit{Write a differential equation for this problem.}\\

\textbf{What's an ODE?} Derivative of stock wrt to time. 
\begin{align*}
    \frac{d y(t)}{dt} = - \alpha
\end{align*}

\subsection{b}
\textit{If there were initially 500 million barrels of oil, then solve for the amount of oil at arbitrary time $T$.}\\

\textbf{Intuition:} This is the same as problem 1b. We need to get the stock level in time period T , which is $y(T)$. To get this, we
could integrate the changes in the stock through time. 

\subsection{c}
\textit{If $\alpha = 2.5$, then when will 75\% of the oil in the well be used up? Note from AC: continue to assume $y(0) = 500$.} \\

\textbf{Solution intuition:} We want to know when 500 minus the changes in reserve is 25\% of the original amount. 

Take the ODE and do separation of variables. Which can work with whichever variable we want (\textit{i.e.} $dy$ or $dt$). Because we're interested in the time period that 75\%, I'm going to use $dt$. 
\begin{align}
    \frac{d y(t)}{dt} = - \alpha \\
    d y(t) = - \alpha dt\\
    \int_0^T dy(t) = \int_0^T - \alpha dt 
\end{align}
At this stage, we can see that the change in $y$ from 0 to $T$ is equal to $\int_0^T - \alpha dt $
\begin{align}
    \Delta y &= \int_0^T - \alpha dt\\
    &= - \alpha T
\end{align}

So now we can set up our intuitive problem where we have 25\% is left and we see how many time steps it takes to get there. 
\begin{align*}
    0.25 y(0) = y(0) + \Delta y\\
    0.25 * 500 = 500 - \alpha T
\end{align*}
Solve for $T$. 

\end{document}