\documentclass{article}

\usepackage[paper=letterpaper,margin=2.5cm]{geometry} % Set Margins

%% Math and math fonts
\usepackage{amsmath, amsthm, amssymb, amsfonts}
\usepackage{bbm} % for \mathbbm{1}

% date
\usepackage[nodayofweek]{datetime}

% Color
\usepackage{color, xcolor}

% Misc
\usepackage{environ}  % \collect@body in asmmath
\usepackage{graphicx} % \includegraphics options
\usepackage{mdframed} % text boxes
\usepackage{indentfirst} % Indent first paragraph after section header
\usepackage[shortlabels]{enumitem} % Control enumerate items with [(a)]
\usepackage{comment} % Comments
\usepackage{fancyhdr} % Headers and footers

% Tables
\usepackage{array}

% Sub-figures and figure placement
\usepackage{caption}
\usepackage{subcaption}
\usepackage{float} 

% Graphing
\usepackage{pgfplots}
\pgfplotsset{compat=1.17}
\usepackage{tikz}

% Title Placement
\usepackage{titling}
\setlength{\droptitle}{-6em}

%set indent to 
\setlength{\parindent}{0pt}

% Hyper refs
\usepackage{hyperref}
\hypersetup{
    colorlinks=true,
    linkcolor=blue,
    urlcolor  = blue,
    filecolor=magenta,      
    urlcolor=blue,
    citecolor = blue,
    anchorcolor = blue
}

% % Citation management
\usepackage{natbib}
\bibliographystyle{abbrvnat}
\setcitestyle{authordate,open={(},close={)}}

\pagestyle{fancy}

\usepackage[paper=letterpaper,margin=2.5cm]{geometry} % Set Margins

%% Math and math fonts
\usepackage{amsmath, amsthm, amssymb, amsfonts}
\usepackage{bbm} % for \mathbbm{1}

% date
\usepackage[nodayofweek]{datetime}

% Color
\usepackage{color, xcolor}

% Misc
\usepackage{environ}  % \collect@body in asmmath
\usepackage{graphicx} % \includegraphics options
\usepackage{mdframed} % text boxes
\usepackage{indentfirst} % Indent first paragraph after section header
\usepackage{comment} % Comments
\usepackage{fancyhdr} % Headers and footers

% Tables
\usepackage{array}

% Sub-figures and figure placement
\usepackage{caption}
% \usepackage{subcaption}
\usepackage{float} 

% Graphing
\usepackage{pgfplots}
\pgfplotsset{compat=1.17}
\usepackage{tikz}

% Title Placement
\usepackage{titling}
\setlength{\droptitle}{-6em}

%set indent to 
\setlength{\parindent}{0pt}

% Hyper refs
\usepackage{hyperref}
\hypersetup{
    colorlinks=true,
    linkcolor=blue,
    urlcolor  = blue,
    filecolor=magenta,      
    urlcolor=blue,
    citecolor = blue,
    anchorcolor = blue
}

% % Citation management
\usepackage{natbib}
\bibliographystyle{abbrvnat}
\setcitestyle{authordate,open={(},close={)}}

\newcolumntype{M}{>{$}c<{$}} % Define a new column type for math mode


% ----------------------------------------
% TITLE
% ----------------------------------------

\pagestyle{fancy}

\lhead{Creel}
\chead{Linar Algebra, cont}
\rhead{AMES}

\title{AMES Section Notes -- Week 10, Setting Up population Growth}
\author{Andie Creel}

\begin{document}
\maketitle



\section{Pset 1 Q1}
Consider a population where the population level can be modeled: 
\begin{align}
    N_t = (b-d)N_{t-1}
\end{align}

And survival rate is:
\begin{align}
    S_ij = 1 -d
\end{align}
Where $S_{ij}$ is the survival rate of individuals $i$ years old make it to $j$ years old.\\

Everyone dies after 3 years. \\

We can model this as:
\begin{align}
    \overset{t+1}{N_{0_{t+1}}} &= \overset{t}{b_0 N_{0_t}} + b_1  N_{1_t}+ b_2  N_{3_t} + b_3  N_{3_t}\\
    N_{1_{t+1}} &= S_{0,1} N_{0_t} + 0 + 0 + 0\\
    N_{2_{t+1}} &= 0+ S_{1,2} N_{1_t}+ 0 + 0\\
    N_{3_{t+1}} &= 0 + 0 + S_{2,3} N_{2_t} + 0
\end{align}

Where we're looking at the number of individuals that are $m$ years old in time period $t+1$, $N_{m_{t+1}}$.\\

Can we write this more clearly with matrices?

\begin{align}
    \begin{bmatrix}
        N_{0_{t+1}}\\
        N_{1_{t+1}} \\
        N_{2_{t+1}} \\
        N_{3_{t+1}}\\
    \end{bmatrix} &= 
    \begin{bmatrix}
        b_0 & b_1 & b_2 & b_3\\
        S_{0,1} & 0 & 0 &0 \\
        0 & S_{1,2} & 0 &0 \\
        0 & 0 & S_{2,3} &0 \\
    \end{bmatrix}
    \begin{bmatrix}
        N_{0_{t}}\\
        N_{1_{t}} \\
        N_{2_{t}} \\
        N_{3_{t}}\\
    \end{bmatrix} 
\end{align}

We can then rewrite this in matrix notation by naming these matrices, 
\begin{align}
    \underset{4 \times 1}{N_{t+1}} &= \underset{4 \times 4}{A} \underset{4 \times 1}{N_t}
\end{align}



\end{document}
